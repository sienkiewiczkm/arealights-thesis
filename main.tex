\documentclass[pl]{minipw/minipw}

\usepackage{babel} % https://tex.stackexchange.com/a/166775
\usepackage{polski}
\usepackage{xltxtra}
\usepackage{siunitx}
\usepackage{enumitem}
\usepackage{diagbox}

\usepackage[toc,page]{appendix}
\renewcommand{\appendixtocname}{Załączniki}
\renewcommand{\appendixpagename}{Załączniki}

\usepackage{hyperref}
\usepackage{amsmath} % text in math
\usepackage{amssymb} % extended math symbol collection (Cap, Cup etc.)
\usepackage{listings} % code listings (lstlisting)
\usepackage{subfiles} % subfiles and selective compilation
\usepackage[disable,textsize=small]{todonotes}
\usepackage{fontspec}
\usepackage{fontawesome}
\usepackage{wrapfig} % wrapfigure enviroment
\usepackage{subcaption} % subfigures
\usepackage{tabularx}
\usepackage{physics}
\usepackage{siunitx}

\usepackage{color}

\usepackage[normalem]{ulem}

\usepackage{tikz}
\usepackage{tikz-3dplot}
\usetikzlibrary{arrows.meta,angles,quotes}
\usetikzlibrary{intersections}
\usetikzlibrary{babel}

\usepackage{biblatex}
\addbibresource{source/references.bib}

\usepackage{graphicx}
\graphicspath{ {illustrations/} }

% Code listing configuration
\lstset{
  basicstyle=\small\ttfamily,
  columns=fullflexible,
  keepspaces=true,
  numbers=left,
  frame=lines
}

%-------------------------------------------------------------------------------
% Konfiguracja klasy `minipw`
\setboolean{lady}{false} % kobiety wpisują true, mężczyźni - false
\title{Area lights in computer games}
\titleaux{Światła powierzchniowe w grach komputerowych}
\type{magisters}
\discipline{Informatyka}
\specjal{Projektowanie systemów CAD/CAM}
\author{Kamil Sienkiewicz}
\album{254210}
\supervisor{dr~inż. Joanna Porter}
\date{2017}
\klucze{
    grafika trójwymiarowa czasu rzeczywistego, 
    światła powierzchniowe,
    oświetlenie, 
    oświetlenie bazowane na zjawiskach fizycznych,
    liniowo transformowane konsinusy,
    radiometria
}
\keywords{
    real-time 3D graphics,
    area lights,
    lighting,
    physically based rendering,
    physically based shading,
    linearly transformed cosines,
    radiometry
}
%-------------------------------------------------------------------------------

\sisetup{
  per-mode = fraction,
}

\begin{document}
\sloppy
\setcounter{page}{1}

\begin{streszczenie}
Praca magisterka wprowadza potrzebną teorię do zrozumienia prac naukowych, omawiających tematykę świateł powierzchniowych oraz przedstawia kilka metod generowania obrazu z oświetleniem tego typu. Wytłumaczone zostają podstawowe pojęcia optyki i radiometrii w kontekście grafiki komputerowej, następnie bazujące na nich modele cieniowania nazywanymi modelami opartymi na zjawiskach fizycznych. Przedstawione metody symulacji obejmują uproszczenie odległych lub niewielkich źródeł powierzchniowych punktowymi oraz technikę przybliżenia większych chmurami świateł punktowych. Przypomniana została metoda Monte-Carlo, która, z odpowiednimi usprawnieniami, może zostać potraktowana jako metoda referencyjna do dalszych badań. Wprowadzono również technikę liniowo transformowanych kosinusów, działającą w czasie rzeczywistym i dającą bardzo dobre rezultaty. Do tekstu pracy dołączona jest aplikacja realizująca omówione algorytmy. Struktura i architektura załączonych aplikacji jest opisana w pracy.
\end{streszczenie}

\begin{abstract}
Master thesis introduces necessary theory required to understand authoritative reports discussing the subject of area lights and presents a few methods of image generation with lighting of this type. Basic optics and radiometry concepts have been explained in context of computer graphics. Physically based rendering models have been derived from mentioned physics laws. The presented simulation methods include far or small area light approximation by point light and general technique for approximating bigger area lights with point light clusters. Monte-Carlo algorithm has been recalled, which with some adjustments is suitable for the reference method. Linearly transformed cosines method has been explained. This method is suitable for real-time and gives very good resources. This thesis includes an example applications implementing discussed algorithms. The structure and architecture of this application is included.
\end{abstract}

\makestatement

\cleardoublepage
\tableofcontents

%\listoftodos

\cleardoublepage
\pagestyle{fancy}

\subfile{source/notation_table}
\subfile{source/ch01_introduction}
\subfile{source/ch02_pbr}
\subfile{source/ch03_montecarlo}
\subfile{source/ch04_ltc}
\subfile{source/ch05_results}
\subfile{source/ch06_architecture}
\subfile{source/ch07_summary}

\cleardoublepage
\begin{appendices}
\subfile{source/appendix_pdf_maxima}
\end{appendices}

\printbibliography

\end{document}
