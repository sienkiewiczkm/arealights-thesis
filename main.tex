\documentclass[pl]{minipw/minipw}

\usepackage{babel} % https://tex.stackexchange.com/a/166775
\usepackage{polski}
\usepackage{xltxtra}
\usepackage{siunitx}
\usepackage{enumitem}

\usepackage[toc,page]{appendix}
\renewcommand{\appendixtocname}{Załączniki}
\renewcommand{\appendixpagename}{Załączniki}

\usepackage{hyperref}
\usepackage{amsmath} % text in math
\usepackage{amssymb} % extended math symbol collection (Cap, Cup etc.)
\usepackage{listings} % code listings (lstlisting)
\usepackage{subfiles} % subfiles and selective compilation
\usepackage[disable,textsize=small]{todonotes}
\usepackage{fontspec}
\usepackage{fontawesome}
\usepackage{wrapfig} % wrapfigure enviroment
\usepackage{subcaption} % subfigures
\usepackage{tabularx}
\usepackage{physics}
\usepackage{siunitx}

\usepackage{color}

\usepackage[normalem]{ulem}

\usepackage{tikz}
\usepackage{tikz-3dplot}
\usetikzlibrary{arrows.meta,angles,quotes}
\usetikzlibrary{intersections}
\usetikzlibrary{babel}

\usepackage{biblatex}
\addbibresource{source/references.bib}

\usepackage{graphicx}
\graphicspath{ {illustrations/} }

% Code listing configuration
\lstset{
  basicstyle=\small\ttfamily,
  columns=fullflexible,
  keepspaces=true,
  numbers=left,
  frame=lines
}

%-------------------------------------------------------------------------------
% Konfiguracja klasy `minipw`
\setboolean{lady}{false} % kobiety wpisują true, mężczyźni - false
\title{Area lights in computer games}
\titleaux{Światła powierzchniowe w grach komputerowych}
\type{magisters}
\discipline{Informatyka}
\specjal{Projektowanie systemów CAD/CAM}
\author{Kamil Sienkiewicz}
\album{254210}
\supervisor{dr~inż. Joanna Porter}
\date{2017}
\klucze{
    grafika trójwymiarowa czasu rzeczywistego, 
    światła powierzchniowe,
    oświetlenie, 
    oświetlenie bazowane na zjawiskach fizycznych,
    liniowo transformowane konsinusy
}
\keywords{
    real-time 3D graphics,
    area lights,
    lighting,
    physically based rendering,
    physically based shading,
    linearly transformed cosines
}
%-------------------------------------------------------------------------------

\sisetup{
  per-mode = fraction,
}

\begin{document}
\sloppy
\setcounter{page}{1}

\begin{streszczenie}
Wykorzystanie świateł powierzchniowych wyraźnie zwiększa realizm generowanej grafiki komputerowej.

Praca magisterka omawia kilka metod generowania oświetlenia pochodzącego ze świateł powierzchniowych oraz wprowadza potrzebną teorię do zrozumienia tematu.

Wprowadzone zostają pojęcia związane z oświetleniem bazowanym na zjawiskach fizycznych, omówione są podstawowe pojęcia optyki, radiometrii i kolorymetrii. Omówione zostało uproszczenie odległych lub niewielkich źródeł światła światłami punktowymi. Pierwszą omówioną metodą jest przybliżenie większego światła powierzchniowego chmurą świateł punktowych.

Przypomniana została metoda Monte-Carlo w kontekście generowania grafiki trójwymiarowej wliczając w to światła powierzchniowe. Metoda ta może zostać potraktowana jako metoda referencyjna do dalszych badań.

Kolejno została przedstawiona metoda liniowo transformowanych kosinusów działająca w czasie rzeczywistym i dająca bardzo dobre rezultaty.

Do tekstu pracy dołączona jest aplikacja realizująca omówione algorytmy. Struktura i architektura aplikacji jest opisana w pracy.
\end{streszczenie}

\begin{abstract}
\textit{Przetłumaczyć po zaakceptowaniu polskiej wersji.}
\end{abstract}

\makestatement

%\cleardoublepage
\tableofcontents

%\listoftodos

%\cleardoublepage
\pagestyle{fancy}

\subfile{source/notation_table}
\subfile{source/ch01_introduction}
\subfile{source/ch02_pbr}
\subfile{source/ch03_montecarlo}
\subfile{source/ch04_ltc}
\subfile{source/ch05_results}
\subfile{source/ch06_architecture}
\subfile{source/ch07_summary}

\begin{appendices}
\subfile{source/appendix_pdf_maxima}
\end{appendices}

\printbibliography

\end{document}
