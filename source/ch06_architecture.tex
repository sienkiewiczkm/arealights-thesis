\documentclass[../main.tex]{subfiles}

\newcommand{\graphvizscale}{0.09}

\begin{document}

\chapter{Program \texttt{ltc\_fitter}}
\label{appendix:ltc_fitter}

Program \texttt{ltc\_fitter} służy do generowania przekształceń przybliżających rozkłady opisane przez modele BRDF za pomocą metody liniowo transformowanych rozkładów kosinusowych. Zadaniem programu jest wyznaczenie współczynników oraz zapisanie tablic do podglądu wartości (ang. \textit{lookup tables}) do pliku.

Kod programu można znaleźć na dołączonej płycie CD lub w repozytorium online pod adresem: \url{https://github.com/sienkiewiczkm/ltc_fitter}\footnote{W przypadku skorzystania z repozytorium online nie można zapomnieć o pobraniu zależności, które są dołączone w formie podmodułów \url{https://git-scm.com/docs/git-submodule}}.

\section{Opis użytkowania}

Aplikacja \texttt{ltc\_fitter} jest programem wywoływanym z linii poleceń, po uruchomieniu terminala i przejściu do odpowiedniego katalogu możemy uruchomić proces wyznaczenia parametrów z domyślnymi ustawieniami wywołując komendy z listingu \ref{lst:ltc_fitter_default}.

\begin{lstlisting}[
    language=bash,
    numbers=none,
    label={lst:ltc_fitter_default},
    caption={Uruchomienie programu dopasowywującego z domyślnymi parametrami}
]
       (unix) $ ./ltc_fitter
    (windows) $ ltc_fitter
\end{lstlisting}

Ustawienia mogą zostać zmodyfikowane przy użyciu następujących modyfikatorów:
\begin{description}
    \item[\texttt{--brdf=\textit{<method>}}] zmiana referencyjnego modelu BRDF, obecnie wspierany jest tylko model \texttt{ggx}
    
    \item[\texttt{--resolution=\textit{<uint>}}] rozdzielczość podglądu, oznacza liczbę podziałów kąta padania i parametru chropowatości $\alpha$, domyślnie jest równy $64$
    
    \item[\texttt{--min-roughness=\textit{<floating>}}] minimalna wartość parametru chropowatości, zbyt niskie wartości będą skutkować błędami obliczeń, domyślnie $0.001$
    
    \item[\texttt{--output=\textit{<string>}}] nazwa pliku wyjściowego, domyślnie nazwa pliku będzie zbudowana w następujący sposób: \texttt{result\_\textit{<rozdzielczość>}\_\textit{<rozdzielczość>}.ltc}
\end{description}

Przykłady wywołań z niestandardowymi ustawieniami znajdują się na listingu \ref{lst:ltc_fitter_custom_parameters}

\begin{lstlisting}[
    language=bash,
    numbers=none,
    label={lst:ltc_fitter_custom_parameters},
    caption={Uruchomienie programu dopasowywującego z własnymi parametrami}
]
(unix) $ ./ltc_fitter --brdf=ggx --resolution=16 --output=quick16x16.ltc
(unix) $ ./ltc_fitter  --min-roughness=0.01 --output=large128.ltc --resolution=128
\end{lstlisting}

\section{Kompilacja}
\label{section:ltcFitterCompilation}

Aplikacja wykorzystuje pakiet CMake do wygenerowania projektu dla konkretnego kompilatora lub środowiska (np. clang, Visual Studio IDE), który będzie można wykorzystać potem do właściwego zbudowania aplikacji. Dokładny opis użytkowania tego programu można znaleźć w dokumentacji projektu CMake \cite{CMakeDoc}. Przykładowa kompilacja w systemie z rodziny Unix (przy użyciu domyślnego kompilatora C++ dla systemu) może zostać przeprowadzona w sposób przedstawiony na listingu \ref{lst:ltc_fitter_compilation}.

\begin{lstlisting}[
    language=bash,
    numbers=none,
    label={lst:ltc_fitter_compilation},
    caption={Kompilacja programu dopasowywującego}
]
 ~/ltc_fitter       $ mkdir build && cd build
 ~/ltc_fitter/build $ cmake ..
 ~/ltc_fitter/build $ make -j4
\end{lstlisting}

Dla środowiska Windows można wykorzystać nakładkę CMake GUI ułatwiającą wygenerowanie projektu. Przykładowy proces wygenerowania projektu w ten sposób przedstawiony jest na rys. \ref{fig:CMakeGUIUsage}.

Projekt został przetestowany pod kątem błędów kompilacji z użyciem poniższych kompilatorów/środowisk:
\begin{itemize}
    \item gcc 6.0
    \item clang
    \item Visual Studio 2017 Community Edition
\end{itemize}

\begin{figure}
    \centering
    
    \begin{subfigure}[t]{0.40\textwidth}
        \includegraphics[width=\textwidth]{architecture/cmake/gui-step-1}
        \caption{Wybieramy katalog główny kodu źródłowego w którym znajduje się plik \texttt{CMakeLists.txt} oraz katalog w którym zostaną umieszczone pliki projektu. Klikamy ,,Configure''.}
    \end{subfigure}
    \hspace{0.5cm}
    \begin{subfigure}[t]{0.40\textwidth}
        \centering
        \includegraphics[width=\textwidth]{architecture/cmake/gui-step-2}
        \caption{Wybieramy narzędzie dla którego projekt zostanie wygenerowany, tutaj wybraliśmy ,,Visual Studio 2017''.}
    \end{subfigure}

    \hfill \\ 
    \vspace{0.5cm}

    \begin{subfigure}[t]{0.40\textwidth}
        \centering
        \includegraphics[width=\textwidth]{architecture/cmake/gui-step-3}
        \caption{Po wstępnej konfiguracji pojawiają nam się możliwe opcje do ustawienia, pozostawiamy wszystkie domyślne wartości oraz klikamy ponownie ,,Configure'', następnie ,,Generate''}
    \end{subfigure}
    \hspace{0.5cm}
    \begin{subfigure}[t]{0.40\textwidth}
        \centering
        \includegraphics[width=\textwidth]{architecture/cmake/gui-step-4}
        \caption{Projekt został wygenerowany. Po przyciśnięciu ,,Open Project'' powinno otworzyć się IDE w którym możemy skompilować projekt.}
    \end{subfigure}
    
    \caption{Generowanie projektu przy użyciu CMake GUI 3.10.1. Opracowanie własne.}
    \label{fig:CMakeGUIUsage}
\end{figure}


\section{Struktura kodu źródłowego}

W główym katalogu repozytorium znajdziemy poszczególne pliki oraz katalogi zawierające wszystko co jest potrzebne do uruchomienia aplikacji.

\begin{description}
    \item[\texttt{CMakeLists.txt}] \hfill\\ 
    plik opisujący sposób generowania projektów dla poszczególnych IDE 
    
    \item[\texttt{README.md}] \hfill\\ 
    krótki opis aplikacji i podstawowy sposób użytkowania 
    
    \item[\texttt{scripts/}] \hfill\\ 
    skrypty pomocnicze do generowania podglądów i wyszukiwania błędów
    
    \item[\texttt{src/}] \hfill\\ 
    katalog ze źródłami aplikacji
    
    \item[\texttt{dependencies/}] \hfill\\ 
    katalog ze źródłami zależności aplikacji, znajdują się tam źródła biblioteki \texttt{glm} \cite{GLMProject} do operacji na wektorach oraz \texttt{stb} \cite{STBProject} do operacji na mapach bitowych
\end{description}

\noindent W katalogu \texttt{src/} znajdziemy kilka katalogów oraz plików z całą funkcjonalnością aplikacji podzielone ze względu na zastosowanie:

\begin{description}
   
    \item[\texttt{brdf/}] \hfill\\
    definicje BRDF, klasy wewnątrz implementują modele wyznaczające wartość $f_r$ dla danego zbioru parametrów: chropowatości $\alpha$, kierunku obserwacji $\omega_o$ oraz kierunku padania światła $\omega_i$
    
    \item[\texttt{exporters/}] \hfill\\
    klasy transformujące dane z postaci używanej wewnątrz aplikacji do formy optymalnej do przekazania do zapisu przez klasy z katalogu \texttt{files/}
    
    \item[\texttt{files/}] \hfill\\ 
    funkcjonalność wejścia-wyjścia, klasy wewnątrz potrafią wczytywać oraz zapisywać pliki podglądu na niskim poziomie
    
    \item[\texttt{fitting/}] \hfill\\
    zbiór klas organizujących dopasowywanie rozkładu lambertowskiego do danego rozkładu opisanego przez BRDF dla danych parametrów
    
    \item[\texttt{ltc/}] \hfill\\
    klasy pozwalające na przeprowadzanie obliczeń na przekształconych liniowo rozkładach kosinusowych, zawiera także funkcjonalność obliczania błędu między rozkładem LTC, a BRDF
     
    \item[\texttt{numerical/}] \hfill\\
    funkcje numeryczne ogólnego zastosowania: metoda pełzającego sympleksu dla dowolnego wymiaru, generowanie losowych próbek, podstawowe estymatory z funkcją błędu dla ograniczeń
    
    \item[\texttt{plotting/}] \hfill\\
    funkcje generujące podgląd dla BRDF i LTC dla danych parametrów, generowanie podglądów pozwala na szybszą analizę dokładności i naprawianie błędów implementacyjnych
    
    \item[\texttt{tests/}] \hfill\\
    katalog z bardzo prostymi testami dymnymi oraz eksporterem podglądów dla wyników otrzymanych przez \cite{ltc_heitz} 
    
    \item[\texttt{utils/}] \hfill\\
    stałe i funkcje, które nie wpasowały się do pozostałych modułów
\end{description}

\section{Architektura aplikacji}

Sercem programu jest elastyczna implementacja algorytmu pełzającego sympleksu z możliwością uruchomienia go w wariancie z funkcją kary. Diagram klas zawierający kluczowe elementy do zrozumienia architektury dopasowywania funkcji znajduje się na rys. \ref{fig:FunctionSolveClassDiagram}. 

Znalezienie parametrów opowiadającym minimum dla danej funkcji wymaga implementacji interfejsu \texttt{error\_estimator}, klasa ta wyznacza wartość błędu dla danych parametrów. Błąd ten minimalizowany jest w trakcie poszukiwania rozwiązania optymalnego. W przypadku funkcji kary, istnieje konieczność wyznaczenia wartości ograniczeń, które potem są przeliczane na dodatkowy (karny) błąd zastosowany przez estymator \texttt{penalty\_error\_estimator}. 

\begin{figure}[h]
    \centering
    \includegraphics[scale=\graphvizscale]{architecture/ltcfitter/fitting.pdf}
    \caption{Wybrane elementy klas dotyczących procesu optymalizacji funkcji. Opracowanie własne.}
    \label{fig:FunctionSolveClassDiagram}
\end{figure}

W celu wykonania obliczeń koniecznych do znalezienia macierzy $M$ przekształcającej rozkład kosinusowy do postaci bliskiej rozkładowi referencyjnemu, musimy dodać kilka elementów do powyższej struktury. Dzięki interfejsom, dodanie rozszerzeń jest bardzo proste. Pierwszym krokiem, jest implementacja BRDF referencyjnego, w naszym przypadku GGX (rys. \ref{fig:BRDFClassDiagram}).

\begin{figure}[h]
    \centering
    \includegraphics[scale=\graphvizscale]{architecture/ltcfitter/brdf.pdf}
    \caption{Interfejs \texttt{brdf} standaryzujący operacje na modelach BRDF. Opracowanie własne.}
    \label{fig:BRDFClassDiagram}
\end{figure}

Kolejnymi krokami jest implementacja: 
\begin{enumerate}
    \item reprezentacji liniowo transformowanych kosinusów, którą będziemy traktować jak BRDF,
    \item estymator błędu między obecnym przybliżeniem, a BRDF referencyjnym,
    \item kalkulator ograniczeń, które zostaną wykorzystane do nałożenia kary w przypadku zbliżenia się do granic.
\end{enumerate}

Implementacja już istniejących interfejsów (rys. \ref{fig:LTCClassDiagram}) pozwoli nam na wykorzystanie w pełni mechanizmu wyszukiwania minimum funkcji do znalezienia optymalnej transformacji.

\begin{figure}[h]
    \centering
    \includegraphics[scale=\graphvizscale]{architecture/ltcfitter/ltc1.pdf}
    \includegraphics[scale=\graphvizscale]{architecture/ltcfitter/ltc2.pdf}
    \caption{Klasy implementujące wsparcie dla optymalizacji macierzy $M$ dla metody LTC. Opracowanie własne.}
    \label{fig:LTCClassDiagram}
\end{figure}

Procedurę wyznaczania parametrów wejściowych dla BRDF referencyjnego oraz ich zapisu do pliku możemy znaleźć w funkcji \texttt{build\_lookup}. Funkcja ta, dzieli dziedzinę na zadaną liczbę próbek, znajduje przybliżone rozwiązanie początkowe i uruchamia funkcję \texttt{ltc\_fit}, która z kolei buduje drzewo zależności wymagane przez algorytm pływającego sympleksu i znajduje optymalne rozwiązanie dla danych parametrów wejściowych.

\section{Format plików wyjściowych}

Plik wyjściowy został zoptymalizowany w taki sposób, by jego wczytanie do pamięci karty graficznej nie wymagało żadnych modyfikacji danych. Plik ten możemy podzielić na trzy części: nagłówek, teksturę pierwszą i drugą.

Nagłówek składa się z kilku pól opisujących zawarte dane oraz wersję pliku. Jego struktura przedstawiona jest w tabeli \ref{tab:alf_header}.

\begin{table}[h]
    \centering
    \begin{tabular}{|l|l|l|}
        \hline
        Typ & Rozmiar (bit) & Opis \\ \hline 
        \texttt{char[4]} & $4*1=4$ & Ustalona wartość kontrolna równa ,,ALFS'' \\ \hline
        \texttt{short} & $16$ & Numer główny wersji (ang. \textit{major}) \\ \hline
        \texttt{short} & $16$ & Numer dodatkowy wersji  (ang. \textit{minor}) \\ \hline
        \texttt{uint} & $32$ & Liczba próbek ze względu na kąt (ozn. $n$) \\ \hline
        \texttt{uint} & $32$ & Liczba próbek ze względu na chropowatość (ozn. $m$) \\ \hline
    \end{tabular}
    \caption{Struktura nagłówku pliku wynikowego.}
    \label{tab:alf_header}
\end{table}

Zaraz po nagłówku rozpoczynają się dane pierwszej tekstury. Oznaczmy liczbę próbek ze względu na kąt $n$ i liczbę próbek ze względu na chropowatość $m$. Dane pierwszej tekstury zawierają $nm$ rekordów zawierających wynik dla danego kąta i chropowatości o równym rozmiarze. Pozycję $p$ dla danej próbki dla $i$-tego podziału kąta i $j$-tej wartości chropowatości ($0 \leq i < n$, $0 \leq j < m$) opisuje równanie (\ref{eq:alf_position}). Każdy z rekordów na danej pozycji ma strukturę opisaną w tabeli \ref{tab:alf_ts1_value}.

\begin{equation}
  p(i,j) = m * i + j
  \label{eq:alf_position}
\end{equation}

\begin{table}[h]
    \centering
    \begin{tabular}{|l|l|l|}
        \hline
        Typ & Rozmiar (bit) & Opis \\ \hline 
        \texttt{float} & $32$ & Współczynnik $M_0^0$ znalezionej macierzy wynikowej \\ \hline
        \texttt{float} & $32$ & Współczynnik $M_0^2$ znalezionej macierzy wynikowej \\ \hline
        \texttt{float} & $32$ & Współczynnik $M_1^1$ znalezionej macierzy wynikowej \\ \hline
        \texttt{float} & $32$ & Współczynnik $M_2^0$ znalezionej macierzy wynikowej \\ \hline
    \end{tabular}
    \caption{Struktura próbki w teksturze pierwszej.}
    \label{tab:alf_ts1_value}
\end{table}

Dane drugiej tekstury znajdują się od razu po danych pierwszej tekstury. Rekordy ułożone są w identyczny sposób jak w teksturze pierwszej, ale każdy z nich ma inną strukturę, opisaną w tabeli \ref{tab:alf_ts2_value}.

\begin{table}[h]
    \centering
    \begin{tabular}{|l|l|l|}
        \hline
        Typ & Rozmiar (bit) & Opis \\ \hline 
        \texttt{float} & $32$ & Współczynnik $M_2^2$ znalezionej macierzy wynikowej \\ \hline
        \texttt{float} & $32$ & Norma rozkładu wejściowego $n_D$ \\ \hline
        \texttt{float} & $32$ & Współczynnik Fresnela otrzymany podczas wyznaczania przybliżenia $f_D$ \\ \hline
    \end{tabular}
    \caption{Struktura próbki w teksturze pierwszej.}
    \label{tab:alf_ts2_value}
\end{table}

%\begin{table}[]
%    \begin{tabular}{|l|l|l|l|l|l|}
%        \hline
%        \multicolumn{2}{|l|}{\multirow{5}{*}{Header}}          & char{[}4{]} & 4*1=4 & File magic number           & Equal to ALFS \\ \cline{3-6} 
%        \multicolumn{2}{|l|}{}                                 & short       & 16    & Major version               & Equal to 1    \\ \cline{3-6} 
%        \multicolumn{2}{|l|}{}                                 & short       & 16    & Minor version               & Equal to 0    \\ \cline{3-6} 
%        \multicolumn{2}{|l|}{}                                 & uint        & 32    & Number of angle samples     &               \\ \cline{3-6} 
%        \multicolumn{2}{|l|}{}                                 & uint        & 32    & Number of roughness samples &               \\ \hline
%        \multirow{6}{*}{Texture Slot \#1} & \multirow{4}{*}{1} & float       & 32    & Matrix coefficient: M00     &               \\ \cline{3-6} 
%        &                    & float       & 32    & Matrix coefficient: M02     &               \\ \cline{3-6} 
%        &                    & float       & 32    & Matrix coefficient: M11     &               \\ \cline{3-6} 
%        &                    & float       & 32    & Matrix coefficient: M20     &               \\ \cline{2-6} 
%        & \multicolumn{5}{l|}{...}                                                               \\ \cline{2-6} 
%        & Number samples     & \multicolumn{4}{l|}{Same format as above.}                        \\ \hline
%        \multirow{4}{*}{Texture Slot \#2} & \multirow{3}{*}{1} & float       & 32    & Matrix coefficient: M22     &               \\ \cline{3-6} 
%        &                    & float       & 32    & Original BRDF magniture     &               \\ \cline{3-6} 
%        &                    & float       & 32    & Fresnel coefficient         &               \\ \cline{2-6} 
%        & \multicolumn{5}{l|}{...}                                                               \\ \hline
%    \end{tabular}
%\end{table}


\chapter{Program \texttt{arealights}}

Program \texttt{arealights} jest aplikacją graficzną realizującą algorytmy symulujące światła powierzchniowe zamieszczone w tej pracy.

Kod programu można znaleźć na dołączonej płycie CD lub w repozytorium online pod adresem: \url{https://github.com/sienkiewiczkm/arealights}.

\section{Opis użytkowania}

Aplikacja przykładowa posiada bardzo prosty interfejs składający się z okna konfiguracyjnego oraz podglądu (rys. \ref{fig:app_arealights_view}). Zmiana ustawień powoduje natychmiastowe zaaplikowanie wartości i odświeżenie podglądu. Akcje i przypisane do nich klawisze przedstawione są w tabeli \ref{tab:arealights_keybindings}.

\begin{figure}[h]
    \centering
    \includegraphics[width=0.7\linewidth]{architecture/arealights_view}
    \caption{Widok główny aplikacji przykładowej. 1) Widok ustawień 2) Podgląd. Opracowanie własne.}
    \label{fig:app_arealights_view}
\end{figure}

\begin{table}[h]
    \centering
    \begin{tabular}{|l|l|}
        \hline \textbf{Przycisk} & \textbf{Akcja} \\ \hline
        W & ruch kamery do przodu \\ \hline
        S & ruch kamery do tyłu \\ \hline
        A & ruch kamery w lewo \\ \hline
        D & ruch kamery w prawo \\ \hline
        lewy przycisk myszki & obrót kamery \\ \hline
        P & zapis podglądu do pliku \texttt{.png} \\ \hline
        L & zablokowanie lub odblokowanie kamery \\ \hline
    \end{tabular}
    
    \caption{Lista akcji przypisanych do przycisków w aplikacji \texttt{arealights}.}
    \label{tab:arealights_keybindings}
\end{table}

Widok ustawień podzielony jest na sekcje: ustawienia światła (ang. \textit{Light settings}), ustawienia sceny (ang. \textit{Scene settings}) oraz ustawienia zależne od wybranej metody oświetlenia.

Ustawienia światła (oznaczone numerem 1 na rys. \ref{fig:app_arealights_settings}) umożliwiają wybór jednej z opisanych w tej pracy metod oraz ustawienia światła w scenie oraz zdefiniowanie energii całkowitej źródła. W przypadku metody chmury punktów, możemy jeszcze ustawić rozmiar macierzy klastra (3 na rys. \ref{fig:app_arealights_settings}) Parametry sceny umożliwiają ustawienie odpowiedniej chropowatości, wybór rodzaju materiału oraz opcjonalne włączenie mapowania tekstur (2 i 4 na rys. \ref{fig:app_arealights_settings}).

\begin{figure}[h]
    \centering
    \begin{subfigure}[t]{0.40\linewidth}
        \includegraphics[width=\linewidth]{architecture/arealights/full-settings}
        \caption{1. Ustawienia światła, kolejno są to: aktualnie używana metoda, pozycja światła prostokątnego, rozmiar źródła, obrót, energia światła. 2. Ustawienia sceny. Kolejno ustawienia oznaczają: wybór rodzaju materiału (stały i wykorzystujący tekstury), w przypadku stałego również właściwości przewodnictwa oraz parametr szorstkości.}
    \end{subfigure}
    \hspace{0.05\textwidth}
    \begin{subfigure}[t]{0.40\linewidth}
        \includegraphics[width=\linewidth]{architecture/arealights/full-settings-2}
        \caption{W przypadku wyboru tekstury, nie możemy dowolnie ustawiać gładkości powierzchni, pobierana jest ona z tekstury, która może zostać wybrana z listy oznaczonej cyfrą 4.}
    \end{subfigure}

    \caption{Widok ustawień aplikacji przykładowej. Opracowanie własne.}
    \label{fig:app_arealights_settings}
\end{figure}

\section{Kompilacja}

Kompilacja tego projektu przebiega analogiczne do poprzedniego programu. Szczegóły dotyczące budowania aplikacji można znaleźć w dodatku \ref{section:ltcFitterCompilation}.

\section{Struktura kodu źródłowego}

W głównym katalogu repozytorium znajdują się poszczególne pliki i katalogi:

\begin{description}
    \item[\texttt{assets/}] \hfill \\ 
    katalogu zawierający tekstury, tablice podglądu wygenerowane przez \texttt{ltc\_fitter} oraz kod programów cieniujących
    
    \item[\texttt{dependencies/}] \hfill \\ 
    katalog ze źródłami zależności
    
    \item[\texttt{source/}] \hfill \\ źródła aplikacji
    
    \item[\texttt{CMakeLists.txt}] \hfill\\ 
    plik opisujący sposób generowania projektów dla poszczególnych IDE 

    \item[\texttt{README.md}] \hfill\\ 
    krótki opis aplikacji i podstawowy sposób użytkowania 
    
\end{description}

\section{Architektura aplikacji}

Hierarchia klas w aplikacji \texttt{arealights} jest znacznie mniej skomplikowana. Kod źródłowy podzielony jest na kilka głównych klas (wyłączając szkielet aplikacji, który zajmuje się podstawowymi operacjami typu stworzenie okna, odczytywanie urządzeń wejścia itd.).

Klasa \texttt{Application} jest główną klasą programu. Zarządza ona całym procesem: przygotowuje wszystkie moduły zależne, scenę testową oraz zasoby wymagane podczas pracy, konfiguruje kontekst OpenGL, odczytuje wejście z urządzeń i przestawia parametry zależnie od intencji użytkownika. Klasy \texttt{GroundTruth}, \texttt{PointLightCluster}, \texttt{LinearlyTransformedCosines} zawierają kod realizujący odpowiednie metody symulacji świateł powierzchniowych uruchamiane w konkretnym etapie renderowania przez instancję klasy \texttt{Application}.

W klasie \texttt{Interface} znajduje się definicja interfejsu użytkownika zrealizowanego za pomocą darmowej biblioteki ImGui \cite{ImGui}. Parametry są przekazywane przez główny obiekt programu do odpowiednich modułów.

Zaimplementowane metody nie współdzielą dużej części logiki, dlatego aplikacja nie zawiera wielu warstw abstrakcji i w dużym stopniu każda z metod zarządza kontekstem renderowania samodzielnie. Zbyt wczesne dodanie dodatkowych warstw mogłoby utrudnić dodawanie nowych metod w przyszłości ze względu na ich inne wymagania.

\end{document}