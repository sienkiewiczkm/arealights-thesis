\documentclass[../main.tex]{subfiles}

\begin{document}
\chapter*{Wstęp}

W początkach lat 2000 najpopularniejszym modelem wykorzystywanym w grach był
model Blinna-Phonga, który nie wymagał dużej mocy obliczeniowej do oświetlenia
nawet złożonych scen.

Model Blinna możemy opisać wzorem:

$$
I(p) = k_a I_a + \sum_{m \in L} \left( {
    k_d (L_m \cdot N) +
    k_s (R_m \cdot V)^{\alpha} i_{m,s}
} \right)
$$

Jak widać model bierze pod uwagę poniższe stałe opisujące materiał:
- $k_a$ - współczynnik odbicia światła otoczenia
- $k_d$ - współczynnik odbicia światła rozproszonego
- $k_s$ - współczynnik odbicia światła lustrzanego

Powyższe stałe zwykle zachowane są w postaci tekstury i wykorzystane prez kartę
graficzną przy cieniowaniu konkretnego punktu.

%\section{PBR - raz dla każdego oświetlenia}

Jednym z problemów prostych modeli oświetlenia jest to, że bardzo często nie
wyglądają one naturalnie w różnych warunkach oświetlenia.

Jak widać, kolor odbicia zależy w bardzo dużym stopniu od artysty i jego
wyobrażenia o otoczeniu. Czasem obiekt, dla którego stworzone są tekstury
wygląda bardzo dobrze w jednym otoczeniu, by w drugim wyglądać nienaturalnie.

Zbudowany model może bardzo dobrze wyglądać na zewnątrz pośród trawy, lecz
w słabo oświetlonej jaskini nie  będzie wyglądał naturalnie.

Artysta nie musi zgadywać jaki kolor będzie odbity

Podejście bazowane na fizyce umożliwia ominięcie etapu "wyobrażenia" jak to ma
wyglądać, a jedynie opisać materiał z którego zbudowany jest obiekt, a resztę
pozostawić równaniom.

%\section{Podstawowe zasady PBR}

Kolejnym problemem prostych modeli jest brak zachowania podstawowych zasad
fizycznych, a mianowicie:

brak zachowania energii, punkt potrafi odbić więcej energii niż do niego
dociera (czynnik odpowiedzialny za odbicie rozproszone nie jest związane
z odbiciem lustrzanym)

brak zachowania symetrii, tzn. inna ilość energii jest odbita po zamianie
kierunku obserwacji i kierunku padania wiązki światła.

Z tych powodów rozpoczęto poszukiwanie modeli możliwych fizycznie (*plausible*).

Zbudowano modele takie jak:
- General Trowbridge-Reitz (GTR)
- jakies inne?

W tych modelach jednak, ze względu na koszt obliczeniowy światła i w tych
modelach pozostały teoretyczne źródła punktowe.

Do uzyskania fotorealistycznego rezultatu nie wystarczy skompresować świateł do
jednego punktu.

Naturalne światła występujące w przyrodzie zawsze mają pewną niezerową
powierzchnię.

Budując fizycznie poprawny model, ważnym jest, aby uwzględnić kształt źródła
światła (jego powierzchnię).

Wiele źródeł sugeruje, że artyści budujący sceny w grach unikają stosowania
materiałów o dużej refleksywności ze względu na nienaturalne odbicia źródeł
punktowych \cite{pbr_frostbite} \cite{pbr_ue4}.

Do uzyskania większej jakości obrazu wymagane jest porzucenie tego przybliżenia
i opracowanie techniki umożliwiającej obliczenie wpływ światła na obserwatora
wyemitowanego przez światło o pewnej powierzchni.

Przetestowane metody:
- przybliżenie powierzchni wieloma światłami punktowymi
- przyrostowa metoda Monte-Carlo
- liniowo transformowane cosinusy
- punkt reprezentatywny (Unreal)

\end{document}
