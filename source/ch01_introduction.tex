\documentclass[../main.tex]{subfiles}

\begin{document}
\chapter*{Wstęp}

Wszystkie źródła światła w naszym świecie takie jak słońce, żarówki, świetlówki, latarni czy ekrany naszych sprzętów elektronicznych posiadają pewną skończoną powierzchnię. W zależności od rozmiaru, kształtu źródła oraz intensywności wpływają one inaczej na estetykę i nasz odbiór danej sceny. Nie bez znaczenia są również materiały użyte w otoczeni. Lustra lub bardzo gładkie powierzchnie, takie jak metal lub lakierowane panele podłogowe, bardzo dobrze odwzorowują właściwości geometryczne źródeł. Obecnie wykorzystywane techniki czasu rzeczywistego niestety pomijają ten aspekt ze względów praktycznych. Takie uproszczenie uniemożliwia nam wykorzystania pełnego wachlarza gładkości, ponieważ ,,oszustwa'' wykorzystane po drodze mogłyby być bardzo łatwo wychwycone przez człowieka. Niestety, takie istnienie takiego ograniczenia sprawia, że nie jesteśmy w stanie precyzyjnie symulować naszego świata. Pewne materiały są dla nas niedostępne, ich wykorzystanie zburzy całkowicie iluzję rzeczywistości. Z tego powodu, firmy oraz entuzjaści zajmującą się grafiką komputerową poświęcają coraz więcej czasu na opracowywanie metod pozwalających na zastosowanie świateł powierzchniowych również w grach i aplikacjach czasu rzeczywistego. 

Modelowanie zachowania światła w scenie jest najistotniejszym aspektem każdego systemu renderowania niezależnie od końcowego zastosowania, to właśnie symulacja oświetlenia jest produktem końcowym, który jest oceniany przez użytkownika. Twórcy aplikacji czasu rzeczywistego przez bardzo długi czas nie mogli sobie pozwolić na wykorzystanie nawet średnio zaawansowanych technik symulacji oświetlenia ze względu na znaczące ograniczenia techniczne. Rozwój sprzętu komputerowego, szczególnie kart graficznych, pozwala nam jednak na wykorzystanie coraz bardziej wymagających metod bez poświęcania płynności aplikacji.

W pierwszym rozdziale tej pracy zostaną omówione podstawowe pojęcia związane z zachowaniem wiązek światła w naszym świecie oraz metodami analizy i modelowania używanymi obecnie przez zarówno przemysł filmowy oraz gier komputerowych. Kolejno przedstawiam najpowszechniejsze modele opisujące interakcję światła z powierzchnią wykorzystywanymi w grach komputerowych. W tym rozdziale przedstawiona jest również pierwsza metoda aproksymacji świateł powierzchniowych chmurą świateł punktowych.

W drugim rozdziale przypomniane zostały podstawy metody Monte-Carlo oraz wprowadzone wydajne warianty biorące pod uwagę rozkład prawdopodobieństwa zdarzeń. Na podstawie tego przedstawimy metodę wykorzystującą algorytm Monte-Carlo, którą może służyć jako model referencyjny. Jest to metoda bazowana bezpośrednio na właściwościach świateł powierzchniowych przez wynik jest bardzo zbliżony do idealnego przy odpowiednio dużej ilości kroków. Metody tego typu wykorzystywane są w przemyśle filmowym, gdzie ograniczenia czasowe oraz zasoby obliczeniowe nie grają tak istotnej roli. Niestety, na obecnych domowych komputerach takie metody nie mogą zostać z powodzeniem w pełni wykorzystane w aplikacjach czasu rzeczywistego. 

Trzeci rozdział zawiera opis metody liniowo transformowanych konsinusów. Jest to obiecująca metoda, która zamiast rozwiązywać oryginalne równania całkowe skupia się na podobnych wyrażeniach ktore posiadają wzór jawny. Wykorzystanie tego typu metody nie powoduje dużych ograniczeń, a metoda daje realistyczne rezultaty. W rozdziale opisuję sposób wyznaczenia odpowiedniego przybliżenia oraz wykorzystanie rezultatów do cieniowania sceny.

Czwarty rozdział zawiera rezultaty uzyskane w aplikacji dołączonej do pracy oraz porównanie zbadanych metod. Dołączone są wizualne porównania jak i średnie czasy wygenerowania danej klatki obrazu.

Piąty rozdział opisuje metodę budowania aplikacji przykładowych, ich sposób obsługi oraz architekturę kodu źródłowego.

\end{document}
