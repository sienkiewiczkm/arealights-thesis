\documentclass[../main.tex]{subfiles}

\begin{document}
\chapter*{Wstęp}

Podstawowym budulcem każdej sceny są źródła światła, bez nich niezależnie od tego co w niej umieściliśmy, nie zobaczylibyśmy kompletnie nic. Wszystkie źródła światła w naszym świecie posiadają pewne właściwości fizyczne, między innymi moc emisji, spektrum emitowanego światła, jego rozmiar oraz geometrię. W zastosowaniach grafiki komputerowej czasu rzeczywistego bardzo dużo z tych właściwości jest pomijanych ze względu na ograniczenia techniczne. Jedną z najczęsciej pomijanych właściwości jest geometria, a za tym powierzchnia takiego światła.

Wszystkie źródła światła w naszym świecie posiadają pewną skończoną powierzchnię, która emituje światło podróżujące po scenie, by w końcu trafić do naszego oka. W zależnośći od rozmiaru i odległości takiego źródła wpływają one inaczej na nasz odbiór otoczenia. Pomieszczenia oświetlone świetłówkami, lub ekranami elektronicznymi będą wyglądać zupełnie inaczej niż takie, które zostało wyposażone w żarówki o równej mocy całkowitej.

Obecnie stosowane przybliżenia podczas generowania grafiki zakładają puntkową naturę źródła światła, która jest bardzo łatwo dostrzegalna w większości scen. Stosowanie takich świateł wprowadza wiele ograniczeń w zbiorze materiałów, które możemy wykorzystać. Bardzo gładkie, prawie lustrzane powierzchnie często mogłoby zburzyć iluzję tego, że większe źródła światła faktycznie pasują do otoczenia bez dodatkowych ,,oszustw''.

Takie ograniczenie sprawia, że dopóki będziemy wkorzysytwać metody, które pomijają tą właściwość nigdy nie będziemy w stanie na prawdę opisywać naszego świata w dowolnych warunkach. Powierzchnia światła jest bardzo istotna dla wielu materiałów, takich jak np. gładki marmur podłogowy oświetlony ekranem telewizora.

Modelowanie zachowania światła w scenie jest najistotniejszym aspektem każdego systemu renderowania niezależnie od końcowego zastosowania, to właśnie symulacja oświetlenia jest produktem końcowym, który jest oceniany przez użytkownika. Drobne błędy w detekcji widocznych obiektów, drobne potknięcia ooptymalizacyjne nie zostaną zauważone tak łatwo jak niewielkie artefakty graficzne. Twórcy aplikacji czasu rzeczywistego przez bardzo długi czas nie mogli sobie pozwolić na wykorzystanie nawet średnio zaawansowanych technik symulacji oświetlenia ze względu na znaczne ograniczenia techniczne. Rozwój sprzętu komputerowego, szczególnie kart graficznych, pozwala nam jednak na wykorzystanie coraz bardziej wymagających metod bez poświęcania płynności aplikacji.

Z tych powodów, coraz więcej profejsonalnych firm oraz entuzjastów szuka dobrych metod opisu świateł powierzchniowych możliwych do przeprowadzenia w czasie rzeczywistym. Znalezienie takiej metody pozwoli nam otworzenie kolejnych drzwi, za którymi, moim zdaniem, będzie kolejny ważny etap rozwoju grafiki komputerowej.

W tej pracy przedstawiam teorię niezbędną do zrozumienia prac naukowych opisujących zagadnienia związane z oświetleniem, w którym powierzchnia źródła jest brana pod uwagę. Przedstawiam również metodę referencyjną opartą na metodzie Monte-Carlo oraz dwie wybrane metody możliwe do uruchomienia w czasie rzeczywistym. Obie te metody mają dużo ograniczeń, jednak stanowią ważny krok w poszukiwaniu idealnego rozwiązania.

W pierwszym rozdziale tej pracy zostaną omówione podstawowe pojęcia związane z zachowaniem wiązek światła w naszym świecie oraz metodami analizy i modelowania używanymi obecnie przez zarówno przemysł filmowy oraz gier komputerowych. Kolejno przedstawiam najpowszechniejsze modele bazujące na zjawiskach fizycnych opisujące interakcję światła z powierzchnią, często wykorzystywanymi w grach komputerowych. W tym rozdziale przedstawiona jest również pierwsza metoda aproksymacji świateł powierzchniowych chmurą świateł punktowych.

W drugim rozdziale przypomniane zostały podstawy metody Monte-Carlo oraz wprowadzone wydajne warianty biorące pod uwagę rozkład prawdopodobieństwa zdarzeń. Na podstawie tego wyprowadzona została metodę wykorzystującą algorytm Monte-Carlo, którą może służyć jako model referencyjny. Jest to metoda bazowana bezpośrednio na właściwościach świateł powierzchniowych przez wynik jest bardzo zbliżony do idealnego przy odpowiednio dużej ilości iteracji. Metody tego typu wykorzystywane są w przemyśle filmowym, gdzie ograniczenia czasowe oraz zasoby obliczeniowe nie grają tak istotnej roli. Niestety, na obecnych domowych komputerach takie metody nie mogą zostać z powodzeniem w pełni wykorzystane w aplikacjach czasu rzeczywistego.

Trzeci rozdział zawiera opis metody liniowo transformowanych konsinusów. Jest to obiecująca metoda, która zamiast rozwiązywać oryginalne równania całkowe skupia się na podobnych wyrażeniach ktore posiadają wzór jawny. Wykorzystanie tego typu metody nie wprowadza wielu dodatkowych ograniczeń, a metoda daje dość realistyczne rezultaty. W rozdziale opisuję sposób wyznaczenia odpowiedniego przybliżenia oraz opisuję wykorzystanie rezultatów do cieniowania sceny.

Czwarty rozdział zawiera rezultaty uzyskane w aplikacji dołączonej do pracy oraz porównanie zbadanych metod, zarówno pod względem estetycznym jak i wydajnościowym. W dołączonych tabelach zamieszczone są obrazy poglądowe oraz średnie czasy wygenerowania klatki obrazu danego rodzaju.

Piąty rozdział skupia się na aplikacjach dwóch aplikacjacha dołączonych do pracy. Do pracy został dołączony program realizujący przedstawione metody w tej pracy oraz drugi, który wyznacza parametry dla jednej z metod eksportowane potem w formie pliku. Omówiłem metodę budowania aplikacji ze źródeł, instrukcję użytkowania oraz architekturę kodu źródłowego.

\end{document}
