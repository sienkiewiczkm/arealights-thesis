\documentclass[../main.tex]{subfiles}

\begin{document}
\chapter*{Wstęp}

W grafice trójwymiarowej, niezależnie od zastosowania, do wygenerowania sceny potrzebujemy dwóch kluczowych elementów: sposobu opisu geometrii oraz modelu symulacji oświetlenia. Przez lata zastosowania czasu rzeczywistego nie mogły sobie pozwolić ani na dokładny opis geometrii, ani na rzeczywisty model oświetlenia. Wraz z rozwojem technologii jednak sytuacja ta ulega zmianie i coraz częściej twórcy gier pozwalają sobie na bardziej skomplikowane modele.

Opis geometrii osiągnął bardzo dobry poziom za sprawą dużej siły obliczeniowej kart graficznych - aplikacje czasu rzeczywistego są w stanie obsłużyć wyświetlenie milionów trójkątów na sekundę, co daje jeszcze lepsze rezultaty przy wykorzystaniu technologii takich jak tessalacja. Różnice między generacjami urządzeń pod tym względem nie są już takie oczywiste. Dokładne modele oświetlenia wciąż są wyzwaniem dla obecnego sprzętu, z tego powodu wykorzystywane są bardzo duże przybliżenia z wieloma dodatkowymi założeniami.

W związku z ograniczeniami technicznymi, modele oświetlenia bardzo często pomijały wiele właściwości fizycznych świateł, które utrudniały obliczenia. Z wielu parametrów, np. mocy emisji, spektrum emitowanego światła, rozmiaru i geometrii, w zastosowaniu gier pozostało tylko kilka, dodatkowo uproszczonych.

Wszystkie źródła światła w naszym świecie posiadają pewną skończoną powierzchnię, która emituje światło podróżujące po scenie, trafiające w końcu do naszego oka. W zależności od rozmiaru i odległości takiego źródła wpływają one inaczej na nasz odbiór otoczenia. Pomieszczenia oświetlone świetlówkami lub ekranami elektronicznymi będą wyglądać zupełnie inaczej niż takie, które zostało wyposażone w małe żarówki.

Obecnie stosowane przybliżenia podczas generowania grafiki zakładają punktową naturę źródła światła, która jest bardzo łatwo dostrzegalna w większości scen. Umożliwia to ominięcie bardzo kosztownego całkowania, co jest bardzo kuszące w zastosowaniach czasu rzeczywistego. Stosowanie takich świateł wprowadza wiele ograniczeń w zbiorze materiałów, które możemy wykorzystać. Bardzo gładkie, prawie lustrzane powierzchnie często mogłoby zburzyć iluzję tego, że większe źródła światła faktycznie pasują do otoczenia bez dodatkowych ,,oszustw''.

Takie ograniczenie sprawia, że dopóki będziemy wykorzystywać metody, które pomijają tą właściwość nigdy nie uda nam się naprawdę opisać i w przekonywujący sposób symulować naszego świata w dowolnych warunkach. Powierzchnia światła jest bardzo istotna dla wielu gładkich materiałów, takich jak np. marmur podłogowy oświetlony ekranem telewizora.

Modelowanie zachowania światła w scenie jest najistotniejszym aspektem każdego systemu renderowania niezależnie od końcowego zastosowania.
Symulacja oświetlenia jest właściwym produktem końcowym, który jest oceniany przez użytkownika. Drobne błędy w detekcji widocznych obiektów, czy potknięcia optymalizacyjne nie zostaną zauważone tak łatwo, jak niewielkie artefakty graficzne. Twórcy aplikacji czasu rzeczywistego bardzo długo nie mogli sobie pozwolić na wykorzystanie nawet średnio zaawansowanych technik symulacji oświetlenia, ze względu na znaczne ograniczenia techniczne. Rozwój sprzętu komputerowego, szczególnie kart graficznych, pozwala jednak na wykorzystanie coraz bardziej wymagających metod bez poświęcania płynności aplikacji.

Z tych powodów, coraz więcej profesjonalnych firm oraz entuzjastów szuka dobrych metod opisu świateł powierzchniowych, możliwych do przeprowadzenia w czasie rzeczywistym. Znalezienie takiej metody pozwoli na otworzenie kolejnych drzwi, za którymi, moim zdaniem, będzie kolejny ważny etap rozwoju grafiki komputerowej.

W tej pracy opisuję teorię niezbędną do zrozumienia prac naukowych opisujących zagadnienia związane z oświetleniem, w którym powierzchnia źródła jest brana pod uwagę. Przedstawiam również metodę referencyjną opartą na metodzie Monte-Carlo oraz dwie wybrane techniki możliwe do uruchomienia w czasie rzeczywistym. Obie mają dużo ograniczeń, jednak stanowią ważny krok w poszukiwaniu idealnego rozwiązania.

W pierwszym rozdziale pracy zostaną omówione podstawowe pojęcia związane z zachowaniem wiązek światła w naszym świecie oraz metodami modelowania używanymi obecnie przez zarówno przemysł filmowy oraz gier komputerowych. Kolejno przedstawiam najpowszechniejsze modele bazujące na zjawiskach fizycznych opisujące interakcję światła z powierzchnią, często wykorzystywanymi w grach komputerowych. W tym rozdziale pokazana jest również pierwsza metoda aproksymacji świateł powierzchniowych chmurą źródeł punktowych.

W drugim rozdziale przypomniane zostały podstawy metody Monte-Carlo oraz wprowadzone wydajne warianty biorące pod uwagę rozkład prawdopodobieństwa zdarzeń. Na podstawie tego wyprowadzona została metoda wykorzystująca algorytm Monte-Carlo, która może służyć jako model referencyjny. Jest to metoda oparta bezpośrednio na właściwościach świateł powierzchniowych przez co wynik jest bardzo zbliżony do idealnego przy odpowiednio dużej ilości iteracji. Metody tego typu wykorzystywane są w przemyśle filmowym, gdzie ograniczenia czasowe oraz zasoby obliczeniowe nie grają tak istotnej roli. Niestety, na obecnych domowych komputerach takie metody nie mogą zostać z powodzeniem w pełni wykorzystane w aplikacjach czasu rzeczywistego.

Trzeci rozdział zawiera opis metody liniowo transformowanych kosinusów. Jest to obiecująca metoda, która zamiast rozwiązywać oryginalne równania całkowe skupia się na podobnych wyrażeniach, które posiadają wzór jawny. Wykorzystanie tego typu metody nie wprowadza wielu dodatkowych ograniczeń, a metoda daje dość realistyczne rezultaty. Na końcu przedstawiam sposób wyznaczenia odpowiedniego przybliżenia oraz opisuję wykorzystanie rezultatów do cieniowania sceny.

Czwarty rozdział zawiera rezultaty uzyskane w aplikacji dołączonej do pracy oraz porównanie zbadanych metod, zarówno pod względem estetycznym jak i wydajnościowym. W dołączonych tabelach zamieszczone są obrazy poglądowe oraz średnie czasy wygenerowania klatki obrazu danego rodzaju.

Piąty rozdział skupia się na dwóch aplikacjach dołączonych do pracy. Pierwszą jest program realizujący przedstawione metody w tej pracy, w formie aplikacji graficznej. Drugi wyznacza parametry dla jednej z metod, które są potem eksportowane do pliku wykorzystanego podczas generowania obrazu. Omówiłem metodę budowania aplikacji ze źródeł, instrukcję użytkowania oraz architekturę kodu źródłowego.

\end{document}
