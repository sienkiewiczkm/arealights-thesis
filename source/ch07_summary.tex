\documentclass[../main.tex]{subfiles}

\begin{document}
\chapter{Podsumowanie}

Wykorzystanie świateł powierzchniowych w grach na pewno przyniesie nam znaczną poprawę jakości obrazu, lecz nie istnieją jeszcze uniwersalne techniki, które zadziałają w każdej sytuacji przez co jeszcze będziemy zmuszeni do pozostania w przy światłach punktowych do opisu większości oświetlenia scen. W szczególnych przypadkach, w których cienie nie są wymagane - metoda liniowo transformowanych kosinusów daje rezultaty bliskie wynikom referencyjnym, więc rekomendowałbym ją jako pierwszą metodę, którą powinniśmy się zainteresować.

\section{Kontynuacja pracy}

Aplikację przykładową można rozszerzyć o wiele dodatkowych funkcjonalności. Najprostszymi rozszerzeniami są dodatkowe metody oświetlenia oraz wsparcie dla bardziej skomplikowanych kształtów świateł powierzchniowych takich jak gwiazdy planarne, dyski i inne. Implementacje mogą być również rozszerzone o wsparcie dla kolorowych (teksturowanych) świateł.

Metoda liniowo transformowanych kosinusów może zostać wykorzystana do aproksymacji innych modeli BRDF bez znaczących modyfikacji programu - wystarczy dodać odpowiedni opis modelu i dodać kilka właściwości do kodu startowego.

Obecnie wyszukiwanie parametrów dla macierzy wykorzystanej w metodzie liniowo transformowanych kosinusów jest wykonywane są w jednym wątku, ale specyfika tych obliczeń umożliwia zrównoleglenie dużej części operacji przez co czas wykonania może być skrócony wielokrotnie. Trudniejszym, ale i potencjalnie bardziej efektywnym, podejściem byłoby przeniesienie części wymaganych obliczeń na procesory graficzne z wykorzystaniem technologii takich jak CUDA czy OpenCL.

Istnieje jeszcze kilka metod symulacji świateł powierzchniowych między innymi
metoda punktu reprezentatywnego
\cite{pbr_ue4},
\cite{Drobot}
oraz
różne metody analityczne
\cite{Snyder}
\cite{LecoqAnalyticApproximationsForALS}
.

Otwartym problemem jest również wparcie dla cieni, jednak te zagadnienie jest wielokrotnie trudniejsze niż dla świateł o naturze punktowej.



\end{document}
