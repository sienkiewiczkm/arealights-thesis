\documentclass[../main.tex]{subfiles}

\begin{document}
\chapter{Podsumowanie}

Wykorzystanie świateł powierzchniowych w grach na pewno przyniesie nam znaczną poprawę jakości obrazu, lecz nie istnieją uniwersalne techniki, które zadziałają w każdej sytuacji. Przez pewien czas będziemy jeszcze zmuszeni do pozostania przy światłach punktowych do opisu większości oświetlenia sceny. W wielu przypadkach, w których cienie nie są wymagane, metoda liniowo transformowanych kosinusów daje rezultaty bliskie wynikom referencyjnym, więc rekomendowałbym ją jako pierwszą metodę, którą powinniśmy się zainteresować.

\section{Kontynuacja pracy}

Aplikację przykładową można rozszerzyć o wiele dodatkowych funkcjonalności. Najprostszymi rozszerzeniami są dodatkowe metody oświetlenia oraz wsparcie dla bardziej skomplikowanych kształtów świateł powierzchniowych takich jak gwiazdy planarne, dyski i inne. Implementacje mogą być również rozszerzone o wsparcie dla kolorowych świateł, w tym takich, które są opisane teksturą.

Metoda liniowo transformowanych kosinusów może zostać wykorzystana do aproksymacji innych modeli BRDF bez znaczących modyfikacji programu - wystarczy dodać odpowiedni opis modelu i dodać kilka właściwości do kodu startowego.

Obecnie wyszukiwanie parametrów macierzy przekształcenia w metodzie liniowo transformowanych kosinusów jest wykonywane w jednym wątku, ale specyfika tych obliczeń umożliwia zrównoleglenie dużej części operacji przez co czas wykonania może być skrócony wielokrotnie. Trudniejszym, ale i potencjalnie bardziej efektywnym, podejściem byłoby przeniesienie części wymaganych obliczeń na procesory graficzne z wykorzystaniem technologii takich jak CUDA czy OpenCL.

Implementacja innych algorytmów oświetlenia i porównanie z metodami przedstawionymi w tej pracy może rzucić więcej światła na ten temat. Wartymi uwagi metodami, między innymi, jest metoda punktu reprezentatywnego \cite{pbr_ue4, Drobot} oraz różne metody analityczne \cite{Snyder, LecoqAnalyticApproximationsForALS}.

Otwartym problemem jest również wparcie dla cieni, jednak te zagadnienie jest wielokrotnie trudniejsze niż dla świateł o naturze punktowej i nie istnieją jeszcze żadne dobre propozycje rozwiązujące ten problem.

\end{document}
